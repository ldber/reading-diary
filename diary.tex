\documentclass[12pt, a4paper]{report}
\usepackage{parskip}
\usepackage[utf8]{inputenc}
\usepackage{fancyhdr}
\usepackage{bibentry}
\makeatletter\let\saved@bibitem\@bibitem\makeatother
\usepackage{hyperref}
\makeatletter\let\@bibitem\saved@bibitem\makeatother


\newcommand{\record}[2]{\newpage\section*{\href{/library/#1.pdf}{#1}}\bibentry{#1}\fancyhf{}\lhead{#1}\rhead{#2}\par\bigbreak}
\pagenumbering{gobble}
\pagestyle{fancy}
\title{Reading Diary}
\author{Forename Surname}
\date{\today}

\begin{document}

\begin{titlepage}\maketitle\end{titlepage}
\nobibliography{references.bib}
\bibliographystyle{ieeetr}
\tableofcontents{}


\chapter{System Strength}

\record{ghanavati2016identifying}{25/06/2020}
Prior research has shown that autocorrelation and
variance in voltage measurements tend to increase as power
systems approach instability. This paper seeks to identify the
conditions under which these statistical indicators provide reliable
early warning of instability in power systems. First, the paper
derives and validates a semi-analytical method for quickly calculating the expected variance and autocorrelation of all voltages
and currents in an arbitrary power system model. Building on
this approach, the paper describes the conditions under which
filtering can be used to detect these signs in the presence of measurement noise. Finally, several experiments show which types of
measurements are good indicators of proximity to instability for
particular types of state changes. For example, increased variance
in voltages can reliably indicate both proximity to a bifurcation
and the location of increased stress. On the other hand, growth of
autocorrelation in certain line currents is related less to a specific
location of stress but, rather, is a reliable indicator of stress occurring somewhere in the system; in particular, it would be a clear
indicator of approaching instability when many nodes in an area
are under stress.

\record{gu2019review}{25/06/2020}
Synchronous generators (SGs) are still making major contributions to the re-stabilization of a power system following voltage/frequency disturbances, attributed to their inherent
capability of providing system strength and inertia. However,
SGs powered by fossil fuels are operating to a lesser extent
and scheduled for decommissioning in the National Electricity
Market (NEM) of Australia due to the accelerating increase of
low bidding priced asynchronous generation of wind and solar,
which leads to the reduction and even in some cases, a shortage of
system strength and inertia. This paper comprehensively reviews
the requirements of system strength and inertia in the NEM from
an operational security perspective. Australia is the first country
that established the regulation rules of system strength and
inertia to accommodate issues of an emerging high penetration
level of non-synchronous renewable generation.

\chapter{Example Chapter}

\record{einstein1935can}{27/06/2020}
I can write a summary of the paper here. I can write multiple paragraphs as follows.
\par
The title of this page is the BibTeX key. The title itself is hyperlinked to the PDF of the paper. If the link does not work check the folder structure is consistent and perhaps try with a different PDF viewer (it works in Evince document viewer). When it comes time to cite this paper, I can quickly copy the BibTeX key into my LaTeX file without rummaging through the references.bib file. 
\par
Also note the bibliography entry is displayed above so that I can check the fields in the references.bib file have been entered properly. Since it contains indexable information (e.g. title, authors, year of publication, journal, etc.) we can use the search feature of the PDF viewer to find specific papers.
\par
Each summary begins on a new page and the header also displays the BibTeX key on the left. On the right is the date that I read the paper. There is full control over the order the summaries are displayed by moving them around in the reading.tex file. I like to have a reverse chronological order since what I read recently is at the top.
\par
Each paper can be assigned to a chapter for easy sorting. At the moment it is not possible to add a paper into two chapters simultaneously so I just choose the most appropriate chapter. Later on, I might make a HTML interface that allows assigning tags to each summary to make sorting more practical.
\par
Since this a LaTeX document, it is quite easy to enter math mode to write equations, for example:
$$
\sin^2\theta + \cos^2\theta = 1
$$
It is also possible to have tables and even draw diagrams using TikZ.

\end{document}
